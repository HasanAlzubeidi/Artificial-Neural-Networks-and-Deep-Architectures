\documentclass[a4paper]{article}

\usepackage[swedish]{babel}
\usepackage[latin1]{inputenc}
\usepackage{amssymb}
\usepackage{framed}

\setlength{\parindent}{0pt}
\setlength{\parskip}{3ex}

\begin{document}

\begin{center}
  {\large Artificial Neural Networks and Deep Architectures, DD2437}\\
  \vspace{7mm}
  {\huge Short report on lab assignment 2\\[1ex]}
  {\Large Radial basis functions, competitive learning and self-organisation}\\
  \vspace{8mm}  
  {\Large Author 1, Author 2 and Author 3\\}
  \vspace{4mm}
  {\large September 1, 2018\\}
\end{center}

\begin{framed}
Please be aware of the constraints for this document. The main intention here is that you learn how to select and organise the most relevant information into a concise and coherent report. The upper limit for the number of pages is 6 with fonts and margins comparable to those in this template and no appendices are allowed. \\
These short reports should be submitted to Canvas by the authors as a team before the lab presentation is made. To claim bonus points the authors should uploaded their short report a day before the bonus point deadline. The report can serve as a support for your lab presentation, though you may put emphasis on different aspects in your oral demonstration in the lab.
Below you find some extra instructions in italics. Please remove them and use normal font for your text.
\end{framed}

\section{Main objectives and scope of the assignment}

\textit{List here a concise list of your major intended goals, what you planned to do and what you wanted to learn/what problems you were set to address or investigate, e.g.}\\
Our major goals in the assignment were  
\begin{itemize}
\item to .......
\item to .......
\item to ....... 
\end{itemize}

\textit{Then you can write two or three sentences about the scope, limitations and assumptions made for the lab assignment}\\

\section{Methods}
\textit{Mention here in just a couple of sentences what tools you have used, e.g. programming/scripting environment, toolboxes. If you use some unconventional method or introduce a clearly different performance measure, you can briefly mention or define it here.}\\

\section{Results and discussion - Part I: RBF networks and Competitive Learning \normalsize{\textit{(ca. 2.5-3 pages)}}}

\begin{framed}
\textit{Make effort to be \textbf{concise and to the point} in your story of what you have done, what you have observed and demonstrated, and in your responses to specific questions in the assignment. You should skip less important details and explanations. In addition, you are requested to add a \textbf{discussion} about your interpretations/predictions or other thoughts concerned with specific tasks in the assignment. This can boil down to just a few bullet points or a couple of sentences for each section of your results. \\ Overall, structure each Results section as you like. Analogously, feel free to group and combine answers to the questions, even between different experiments, e.g. with noise-free and noisy function approximation, if it makes your story easier to convey. \\
\\Plan your sections and consider making combined figures with subplots rather than a set of separate figures. \textbf{Figures} have to condense information, e.g. there is no point showing a separate plot for generated data and then for a decision boundary, this information can be contained in a single plot. Always carefully describe the axes, legends and add meaningful captions. Keep in mind that figures serve as a support for your description of the key findings (it is like storytelling but in technical format and academic style. \\
\\Similarly, use \textbf{tables} to group relevant results for easier communication but focus on key aspects, do not overdo it. All figures and tables attached in your report must be accompanied by captions and referred to in the text, e.g. $"$in Fig.X or Table Y one can see ....$"$. \\
\\When you report quantities such as errors or other performanc measures, round numbers to a reasonable number of decimal digits (usually 2 or 3 max). Apart from the estimated mean values, obtained as a result of averaging over multiple simulations, always include also \textbf{the second moment}, e.g. standard deviation (S.D.). The same applies to some selected plots where \textbf{error bars} would provide valuable information, especially where conclusive comparisons are drawn.} 
\end{framed}

\subsection{Function approximation with RBF networks\\ \normalsize{\textit{(ca. 1.5-2 pages)}}}
\textit{Combine results and findings from RBF simulations on both noise-free and noisy function approximation (sin(2x) and square (2x)). Try to organise them into subsections, and please make sure you pay attention to the comparison between RBF- and MLP-based approaches as well as the comparative analyses of batch and on-line learning schemes. Answer the questions, quantify the outcomes, discuss your interpretations and summarise key findings as conclusions.}


We performed function approximation with Radial basis function networks. The functions we approximated were the $sin(2x)$ and $square(2x)$ functions. The training set for the functions was obtained by using the domain $[0:2\pi]$. The step size was 0.1 and we sampled both the functions to obtain the training set. For the test set we started from 0.05 and used the same step size.  



\subsection{Competitive learning for RBF unit initialisation\\ \normalsize{\textit{(ca. 1 page)}}}
\textit{Please refer first to the results in the previous section, i.e. those obtained without any automated initialisation. Then in the next subsection focus on two-dimensional regression with RBF networks.}


\section{Results and discussion - Part II: Self-organising maps \normalsize{\textit{(ca. 2 pages)}}}

\subsection{Topological ordering of animal species}

\subsection{Cyclic tour}

\subsection{Clustering with SOM}

\section{Final remarks \normalsize{\textit{(max 0.5 page)}}}
\textit{Please share your final reflections on the lab, its content and your own learning. Which parts of the lab assignment did you find confusing or not necessarily helping in understanding important concepts and which parts you have found interesting and relevant to your learning experience? \\
Here you can also formulate your opinion, interpretation or speculation about some of the simulation outcomes. Please add any follow-up questions that you might have regarding the lab tasks and the results you have produced.}

\end{document}
